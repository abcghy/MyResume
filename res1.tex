% LaTeX file for resume 
% This file uses the resume document class (res.cls)
\documentclass[11pt]{res} 
\usepackage{ctex}
\usepackage{multicol}
\usepackage{url}
%\usepackage{helvetica} % uses helvetica postscript font (download helvetica.sty)
%\usepackage{newcent}   % uses new century schoolbook postscript font 
\setlength{\textheight}{9.5in} % increase text height to fit on 1-page 

\begin{document}
\name{高惠宇}
\address{  beatbox\_gao@hotmail.com\\ GitHub: \url{https://github.com/abcghy} \\ Blog: \url{https://abcghy.github.io/}}
\address{上海长宁 \\ 17602152878 18818217393}

\begin{resume}

\section{求职意向}
  Android 开发

\section{教育经历}
  上海大学\ 计算机工程与科学\ 本科\ 2012.9\~{}2016.6

\section{语言能力}
  英语 CET 6 \\英语水平良好,经常阅读 GitHub 上的各种文档与 Repo,通过英语查找技术相关资料。也经常上 Youtube 学习一些 conference 和技术分享。

\section{工作经验}
\large 上海卫莎网络科技有限公司 \normalsize\\
\small 2018.10--至今 Android 研发工程师 负责开发马卡龙玩图、绿幕侠。\normalsize

    \emph{马卡龙玩图}\\
    带有自动扣图功能的图片编辑应用,除了图片分割以外,还有融合、换天空、模板、图层混合等功能。 App Store 经常被首页推荐, Android 版也被小米、华为等厂商推荐。\\
    下载链接: \url{https://android.myapp.com/myapp/detail.htm?apkName=com.versa}
\begin{itemize}
\item 使用 Retrofit, LiveData, ViewModel, Room 等重构网络模块以及各种旧代码
\item 负责社区首页、模板、贴纸、融合、换天空、菜单、分享、广告、搜索等核心模块的实现
\item 使用 kotlin 重构部分代码,使代码更加易读,开发更加快速
\item OOM 的探索以及优化,降低用户崩溃概率。使用 AspectJ 对图片进行大小检测,遇到加载错误大小的图片进行提示,减少内存占用。% 图片应用,比较容易内存不足。
\end{itemize}

\emph{绿幕侠}\\
拍摄人物并实时扣掉背景,并辅以炫酷视频为背景,最终合成一个无须绿幕即可扣像的视频应用。用到了 OpenGL、MediaCodec、MediaMuxer 等技术。\\
曾被华为 9x 发布会作为华为芯片的应用app推荐。\\
下载链接: \url{http://app.mi.com/details?id=com.yitian.greenscreenman}\\
华为合作链接:\url{https://developer.huawei.com/consumer/cn/doc/development/hiai-Guides/3142914}
\begin{itemize}
\item 通过 FrameBuffer 实时获取前后摄像头的纹理,输入给本地分割模型获取扣取后的人像纹理,并在此基础上使用商汤SDK做美颜等功能。
\item 配合扣出的人像纹理,融合下载的视频或者用户自己上传的视频,使用 Google 的 Grafika ,将纹理渲染在屏幕上。
\item 由于Android手机的屏幕大小和摄像头支持分辨率繁多的原因,需要对纹理进行裁切。同时用户可以手动移动自己人像的位置,翻转、旋转、复制人像,所以需要做 Android Matrix 和 OpenGL Matrix 的映射。通过限制 Bounding Box 的变化速度,从而使用户更好的对自己的人像进行控制。
\end{itemize}

    \large 上海见寻文化创意有限公司 \emph{概率论}\normalsize\\
  \small 2018.1--2018.10\ Android 研发工程师\ 负责开发52hz公寓、聊天、支付、录音、播放模块。\normalsize
  \begin{itemize}
    \item 独自搭建整个 App 框架,包括网络框架,MVVM 架构。 
    \item 52hz公寓是目前较为流行的一周CP模式。用户可以在公寓里遵循房主的指示与CP进行互动、打卡、聊天;房主可以发布任务,对用户的打卡进行审批。
    \item 聊天模块支持私聊、建群、发送emoji和sticker、发送图片、拍照、重新发送、@别人等功能。底层采用 websocket 与后端进行交互,具有自动重连功能,在应用进入后台时会断开连接,并采用 JPush 接收聊天推送。UI 为自定义 RecyclerView,配合DataBinding、SmartRefreshLayout 达到预期效果。
    \item 失眠旅馆是一个文字、语音交流平台。白天发布较长文字和语音,晚上阅读、收听和评价别人分享。实时将用户的语音录制成AAC格式,并且具备暂停和恢复的功能。
  \end{itemize}

\large 上海海鼎信息工程股份有限公司\normalsize\\
  \small 2017.3--2017.12\ 负责开发千帆掌柜、千帆平板、千帆外勤等项目。分别开发了开单模块;整体 Dialog 架构;库存盘点。\\千帆掌柜下载地址:\url{http://qianfan123.com/#1!dpos}\normalsize
  \begin{itemize}
    \item 千帆掌柜为小型零售店主提供开单、收款、账目统计、管理店铺等功能。具有多种角色,店员也可以进行收款和统计。
    \item 千帆平板是掌柜的升级版,具有全部功能并且可以更好的进行商品管理和开单。
    \item 千帆外勤为巡店人员提供库存盘点、退换货、绩效考核等功能。并且可以轻松查看当日工作情况。
  \end{itemize}

\large 花致信息技术有限公司\normalsize\\
  \small 2016.2--2017.2\ Android 研发工程师\\独立负责开发到了吗用户端 Android app.独立搭建Retrofit+RxAndroid+MVP框架,以便更好地进行单元测试和重构.\\到了吗下载地址: \url{http://www.wandoujia.com/apps/cn.edu.shu.dlm} \normalsize
  \begin{itemize}
    \item 到了吗用户端使用了地图、聊天、支付、推送等模块,为用户提供了方便、快捷的快递服务;
    \item 本人向公司技术开发团队推荐一系列新的开发工具,提高了公司生产力。
  \end{itemize}
 %\clearpage
\section{专业技能}
  \begin{multicols}{3}
    语言
    \begin{itemize}
      \item Java
      \item Kotlin
      \item JavaScript
      \item Python
      \item Go
    \end{itemize}
    框架
    \begin{itemize}
      \item MVVM
      \item Vue
      \item FastAPI
      \item Django
      \item Koa
    \end{itemize}
    技术
    \begin{itemize}
      \item Retrofit
      \item RxJava
      \item OpenGL
      \item Git
      \item Docker
    \end{itemize}
  \end{multicols}

\section{个人评价}
  \begin{enumerate}
      % 整体架构
      \item 熟练使用 DataBinding、ViewModel和LiveData 构建 MVVM 的架构,以及使用 JUnit、Mockito 进行单元测试。
      \item 熟练掌握自定义 view 的方法,并热爱将代码开源至 Github, 之后通过 JitPack 发布可以复用的 Android 组件。
      % 小细节
      \item 在网络框架方面使用 Retrofit,同时使用 RxJava 或协程进行各种异步操作,如获取网络数据、获取数据库数据、定时操作,对其较为了解。
      \item 对 Hotfix 较为了解,在应用上线后,还可以通过 Tinker 来对应用进行热修复。\\ \url{http://abcghy.github.io/2017/02/25/Tinker-Patch-Tutorial/}
      % 反编译等等出彩的地方
      \item 使用 gradle 进行不同环境的配置,更好的进行持续集成开发。
      \item 使用 apktool, dex2jar, jdgui, jadx-gui 进行反编译,分析竞品是如何完善功能,解决问题的。
      % 和 android 关系较轻的
      \item 熟练使用 Docker 搭建各种开发所需环境,如 Jenkins, Gitlab Runner, Nexus 等,独自搭建远程CI/CD环境。
      \item 使用 Git 进行版本控制,熟练掌握 Git Flow,熟练掌握 merge 和 rebase 用法,能够独立解决冲突情况并帮助合作者找出具体问题。
      \item 拥有较强的学习能力,对新事物永远充满好奇。不仅掌握 Android 开发,还对后端、网页前端、iOS 开发有所涉及。
  \end{enumerate}

\end{resume}
\end{document}
