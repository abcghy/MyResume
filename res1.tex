% LaTeX file for resume 
% This file uses the resume document class (res.cls)
\documentclass[11pt]{res} 
\usepackage{ctex}
\usepackage{multicol}
\usepackage{url}
%\usepackage{helvetica} % uses helvetica postscript font (download helvetica.sty)
%\usepackage{newcent}   % uses new century schoolbook postscript font 
\setlength{\textheight}{9.5in} % increase text height to fit on 1-page 

\begin{document}
\name{高惠宇}
\address{  beatbox\_gao@hotmail.com\\ GitHub: \url{https://github.com/abcghy} \\ Blog: \url{http://abcghy.github.io}}
\address{上海杨浦 \\ 17602152878 18818217393}

\begin{resume}

\section{求职意向}
  Android 开发

\section{教育经历}
  上海大学\ 计算机工程与科学\ 本科\ 2012.9\~{}2016.6

\section{语言能力}
  英语 CET 6 \\英语水平良好,经常阅读 GitHub 上的文章、通过英语查找技术相关资料。也经常上 youtube 学习别人在 conference 上传递的经验。

\section{工作经验}
  2018.1--至今\ Android 研发工程师\ 上海见寻文化创意有限公司\\负责开发52hz公寓、聊天、支付、录音、播放模块。
  \begin{itemize}
    \item 独自搭建整个 App 框架,包括网络框架,MVVM 架构。 
    \item 52hz公寓是目前较为流行的一周CP模式。用户可以在公寓里遵循房主的指示与CP进行互动、打卡、聊天;房主可以发布任务,对用户的打卡进行审批。
    \item 聊天模块支持私聊、建群、发送emoji和sticker、发送图片、拍照、重新发送、@别人等功能。底层采用 websocket 与后端进行交互,具有自动重连功能,在应用进入后台时会断开连接,并采用 JPush 接收聊天推送。UI 为自定义 RecyclerView,配合DataBinding、SmartRefreshLayout 达到预期效果。
    \item 失眠旅馆是一个文字、语音交流平台。白天发布较长文字和语音,晚上阅读、收听和评价别人分享。实时将用户的语音录制成AAC格式,并且具备暂停和恢复的功能。
  \end{itemize}
  2017.3--2017.12\ Android 研发工程师\ 上海海鼎信息工程股份有限公司\\负责开发千帆掌柜、千帆平板、千帆外勤等项目。分别开发了开单模块;整体 Dialog 架构;库存盘点。\\千帆掌柜下载地址:\url{http://qianfan123.com/#1!dpos}
  \begin{itemize}
    \item 千帆掌柜为小型零售店主提供开单、收款、账目统计、管理店铺等功能。具有多种角色,店员也可以进行收款和统计。
    \item 千帆平板是掌柜的升级版,具有全部功能并且可以更好的进行商品管理和开单。
    \item 千帆外勤为巡店人员提供库存盘点、退换货、绩效考核等功能。并且可以轻松查看当日工作情况。
  \end{itemize}
  2016.2--2017.2\ Android 研发工程师\ 花致信息技术有限公司\\独立负责开发到了吗用户端 Android app.独立搭建Retrofit+RxAndroid+MVP框架,以便更好地进行单元测试和重构.\\到了吗下载地址: \url{http://www.wandoujia.com/apps/cn.edu.shu.dlm}
  \begin{itemize}
    \item 到了吗用户端使用了地图、聊天、支付、推送等模块,为用户提供了方便、快捷的快递服务;
    \item 本人向公司技术开发团队推荐一系列新的开发工具,提高了公司生产力。
  \end{itemize}
% \clearpage
\section{专业技能}
  \begin{multicols}{3}
    语言
    \begin{itemize}
      \item Java
      \item Kotlin
      \item JavaScript
      \item Python
      \item Go
      \item Swift
    \end{itemize}
    框架
    \begin{itemize}
      \item MVVM
      \item Vue
      \item Django
      \item Koa
    \end{itemize}
    技术
    \begin{itemize}
      \item Retrofit
      \item RxJava
      \item DataBinding
      \item Git
      \item Docker
    \end{itemize}
  \end{multicols}

\section{个人评价}
  \begin{enumerate}
      % 整体架构
      \item 熟练使用 DataBinding、ViewModel和LiveData 构建 MVVM 的架构,以及使用 JUnit、Mockito 进行单元测试。
      \item 熟练掌握自定义 view 的方法,并热爱将代码开源至 Github, 之后通过 JitPack 发布可以复用的 Android 组件。
      % 小细节
      \item 在网络框架方面使用 Retrofit,同时使用 RxJava 进行各种异步操作,如获取网络数据、获取数据库数据、定时操作,对其较为了解。
      \item 对 Hotfix 较为了解,使用 TinkerPatch 使开发人员更快的解决bug。\\ \url{http://abcghy.github.io/2017/02/25/Tinker-Patch-Tutorial/}
      % 反编译等等出彩的地方
      \item 使用 gradle 进行不同环境的配置,更好的进行持续集成开发。
      \item 使用 apktool, dex2jar, jdgui 进行反编译,分析竞品是如何完善功能,解决问题的。
      % 和 android 关系较轻的
      \item 熟练使用 Docker 搭建各种开发所需环境,如 Jenkins,独自搭建远程CI/CD环境。
      \item 使用 Git 进行版本控制,熟悉多人协作开发,能够独立解决冲突情况并帮助合作者找出具体问题。
      \item 拥有较强的学习能力,对新事物永远充满好奇。不仅掌握 Android 开发,还对后端、网页前端、iOS 开发有所涉及。
  \end{enumerate}

\end{resume}
\end{document}