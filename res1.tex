% LaTeX file for resume 
% This file uses the resume document class (res.cls)
\documentclass[11pt]{res} 
\usepackage{ctex}
\usepackage{multicol}
\usepackage{url}
%\usepackage{helvetica} % uses helvetica postscript font (download helvetica.sty)
%\usepackage{newcent}   % uses new century schoolbook postscript font 
\setlength{\textheight}{9.5in} % increase text height to fit on 1-page 

\begin{document}
\name{高惠宇}
\address{  beatbox\_gao@hotmail.com\\ GitHub: \url{https://github.com/abcghy} \\ Blog: \url{http://abcghy.github.io}}
\address{上海宝山 \\  18818217393 17602152878}

\begin{resume}

\section{求职意向}
  Android 开发\\    
  Java 后台开发

\section{教育经历}
  上海大学\ 计算机工程与科学\ 本科\ 2012.9\~{}2016.6

\section{语言能力}
  英语 CET 6 \\英语水平良好,经常阅读 GitHub 上的文章、通过英语查找技术相关资料。也经常上 youtube 学习别人在 conference 上传递的经验。

\section{工作经验}
  2016.2--2017.2\ Android 研发工程师\ 花致信息技术有限公司\\独立负责开发到了吗用户端 Android app.独立搭建Retrofit+RxAndroid+MVP框架,以便更好地进行单元测试和重构.\\到了吗下载地址: \url{http://www.wandoujia.com/apps/cn.edu.shu.dlm}
  \begin{itemize}
    \item 到了吗用户端使用了地图、聊天、支付、推送等模块,为用户提供了方便、快捷的快递服务;
    \item 本人向公司技术开发团队推荐一系列新的开发工具,提高了公司生产力。
  \end{itemize}
  2017.3--2017.12\ Android 研发工程师\ 上海海鼎技术有限公司\\负责开发千帆掌柜、千帆平板、千帆外勤等项目。分别开发了开单模块;整体 Dialog 架构;库存盘点。\\千帆掌柜下载地址:\url{http://qianfan123.com/#1!dpos}
  \begin{itemize}
    \item 千帆掌柜为小型零售店主提供开单、收款、账目统计、管理店铺等功能。具有多种角色转换,店员也可以进行收款
    \item 千帆平板是掌柜的升级版,具有全部功能并且可以更好的进行商品管理和开单。
    \item 千帆外勤为巡店人员提供库存盘点、退换货、绩效考核等功能。并且可以轻松查看当日工作情况。
  \end{itemize}
\clearpage
\section{专业技能}
  \begin{multicols}{3}
    语言
    \begin{itemize}
      \item Java
      \item Kotlin
      \item JavaScript
      \item Python
      \item Swift
      \item PHP
    \end{itemize}
    框架
    \begin{itemize}
      \item MVP
      \item Vue
      \item Spring Boot
      \item Koa
    \end{itemize}
    技术
    \begin{itemize}
      \item Retrofit
      \item RxJava
      \item DataBinding
      \item Git
      \item TinkerPatch
    \end{itemize}
  \end{multicols}

\section{个人评价}
  \begin{enumerate}
      % 整体架构
      \item 熟练使用 DataBinding 和类 MVVM 的架构,以及使用 JUnit、Mockito 进行单元测试。
      \item 使用组件化开发的方式,和 UI 沟通过后,将功能相同的 view 整合成一个组件。熟练掌握自定义 view 的方法,并热爱将代码开源至 Github, 之后通过 JitPack 发布可以复用的 Android 组件。
      % 小细节
      \item 在网络框架方面使用 Retrofit,同时使用 RxJava 进行各种异步操作,如获取网络数据、获取数据库数据、定时操作,对其较为了解。
      \item 对 Hotfix 较为了解,使用 TinkerPatch 使开发人员更快的解决bug。\\ \url{http://abcghy.github.io/2017/02/25/Tinker-Patch-Tutorial/}
      % 反编译等等出彩的地方
      \item 使用 gradle 进行不同环境的配置,更好的进行持续集成开发。
      \item 使用 apktool, dex2jar, jdgui 进行反编译,分析竞品是如何完善功能,解决问题的。
      % 和 android 关系较轻的
      \item 独自建立远程服务器,配置远程编译环境,进行远程编译,节约本地开发资源,提高开发效率。并了解 Jenkins, flow.ci 等 CI 工具的使用。
      \item 使用 Git 进行版本控制,熟悉多人协作开发,能够独立解决冲突情况。
      \item 拥有较强的学习能力,对新事物永远充满好奇。不仅掌握 Android 开发,还对网页前端、iOS 开发, 后端有所涉及。
  \end{enumerate}

\end{resume}
\end{document}